\label[sec]\chap Security for DDS

Nowadays, security is often considered. Technologies for securing communication differs by \glref{TCP}/\glref{IP} layers \cite[www:tcp-ip] - security at Media access layer consists of preventing deterioration of physical media, environmental noise and access to media. At Network layer, \glref{IPsec} (\glref{IP} Security Architecture) protocol is used while Transport layer uses \glref{TLS} (Transport Layer Security) protocol. In this chapter, Application layer security for \glref{DDS} standard \cite[OMG:DDS-SECURITY] and possibilities of implementation in \glref{RTPS} protocol are considered.

\sec Threats

From point of view of \glref{DDS} standard, communication takes place in the domain consisting of participants with various number of publishers and subscribers. In this context, Application layer security threats are following:

\begitems
* Unauthorized\fnote{Difference between authentication and authorization has to be clear. Authentication is verification of (in this context) participant - that the participant is really the one it claims to be. On the other hand, authorization is process of allowing access to data for already authenticated participant.} subscription
* Unauthorized publication
* Tampering and replay
* Unauthorized access to data
\enditems

{\em Unauthorized subscription} is a situation when malicious participant receives data for which it is not allowed to. In network infrastructure where access to media is shared, communication runs over multicast or participants sits on one node, it's practically unavoidable to restrict access to data. The solution is making data unreadable for malicious participant - in other words, applying encryption on publisher's side and sharing keys with authenticated subscribers only.

When malicious participant attempts to send data which it is not allowed to, it's called {\em Unauthorized publication}. For subscriber it's important to receive data only from valid publishers to avoid influence of malicious participant on data. The solution is to include authentication information to data sent by valid publishers so subscribers would be able to recognize data by authenticated publishers from data sent by malicious participant. Authentication of publishers in data can be accomplished by Hash-based message authentication code (\glref{HMAC}) or by digital signature. \glref{HMAC} creates authentication code using secret key shared between publisher and subscriber. Digital signature is based on private/public key pair - authentication code is created as message digest encrypted by private key of publisher. Each subscriber has access to public key of publisher and can use it to decrypt the authentication code to message digest and compare it with message digest calculated by itself. The point is that these two message digests equals if and only if the authentication code is encrypted by publisher's private key and decrypted by publisher's public key. Digital signature is called {\em asymmetric cryptography} (private/public key pair) and is much slower then {\em symmetric cryptography} (shared key), therefore the use of \glref{HMAC} is preferred because of performance reasons.

Valid publisher would send data to subcsriber and malicious participant (in this case, malicious participant will be allowed to subscribe but not to publish). However if the same key is shared between publisher, subscriber and malicious paritcipant, there is no way how to prevent malicious participant to use this shared key for mimicking publisher and sending data to subscriber. This threat is called {\em Tampering and Replay} and can be solved by sharing different keys between publishers and subscribers. When the communication is taken over multicast, multiple \glref{HMAC}s are needed to be included in data, but this solution is still more powerful than using digital signatures.

In the \glref{DDS} network, some participants acts as relay participants forwarding data. These participants need to be trusted as valid publishers and subscribers, but it's not always desirable to let them understand data they work with. The solution for {\em Unauthorized Access to Data} is having different keys for \glref{HMAC} and data encryption and to share keys for decrypting of data only with desired endpoints.

\sec Securing of messages

Securing of messages is application dependent - sometimes it's sufficient to encrypt only user-data, in other applications, submessage's metadata as sequence numbers or writer/reader identifiers are needed to be secured too and in the most secure applications, the whole metatraffic submessages are considered confidential. In order to support of different application scenarios, mechanism called {\em Message Transformation} is introduced. It transforms one \glref{RTPS} message into another \glref{RTPS} message so that the original \glref{RTPS} message or it's submessages may be encrypted into the new one and protected by \glref{HMAC}.

Because of {\em Message Transformation}, new submessages and submessage elements are introduced and the questions about interoperability between secured and non-secured implementations of \glref{RTPS} protocol arises. In implementations of \glref{RTPS} protocol, unknown submessages should be skipped so the regular user-traffic should not be affected, but there is Discovery also. \glref{SPDP} is used by DomainParticipants to discover each other, informations as \glref{IP} address, port, vendor and version are exchanged to bootstrap the communication. Therefore it makes no sense to protect \glref{SPDP} communication, better to use it for exchange of informations needed to bootstrap the secured system. For both - secured and non-secured implementations of \glref{RTPS} protocol, {\em DCPSParticipants} Topic is used in \glref{SPDP} and there is no new secured Topic for \glref{SPDP}.

\glref{SEDP} protocol is used for discovering {\em publishers} and {\em subscribers} of each DomainParitcipant. The {\em DCPSPublications} and {\em DCPSSubscriptions} Topics are used for communication with non-secured endpoints. However for DomainParticipants supporting \glref{DDS} Security, {\em DCPSPublicationsSecure} and {\em DCPSSubscriptionsSecure} Topics and associated DataWriters ({\em SEDPbuiltinPublicationsSecureWriter}, {\em SEDPbuiltinSubscriptionsSecureWriter}) and DataReaders ({\em SEDPbuiltinPublicationsSecureReader}, {\em SEDPbuiltinSubscriptionsSecureReader}) are introduced. These Topics should be used for communication that is considered sensitive.

In \glref{RTPS} protocol, Writer Liveliness Protocol is specified and because data exchange by this protocol could be considered sensitive, \glref{DDS} Security specifies alternate protected way to exchange liveliness information. {\em BuiltinParticipantMessageWriter} and {\em BuiltinParticipantMessageReader} are used to communicate liveliness information with non-secured endpoints. {\em ParticipantMessageSecure} Topic is introduced with associated {\em BuiltinParticipantMessageSecureWriter} and {\em BuiltinParticipantMessageSecureReader}, used to communication liveliness information with endpoints considered sensitive.

Also, there are two completely new builtin Topics:
\begitems
* ParticipantStatelessMessage
* ParticipantVolatileMessageSecure
\enditems

{\em ParticipantStatelessMessage} Topic is used to perform mutual authentication between DomainParticipants. While the mechanism for participant-to-participant communication already exists, it suffers from weakness of reliable protocol - sequence number prediction. HeartBeat messages containing {\em first available sequence number} can be abused by malicious participant to prevent other participants to communicate. Therefore new Topic {\em ParticipantStatelessMessage} with associated {\em BuiltinParticipantStatelessMessageWriter} (Best-Effort StatelessWriter) and {\em BuiltinParticipantStatelessMessageReader} (Best-Effort StatelessReader) is introduced.  

For key exchange between DomainParticipants, reliable and secure communication is needed. On top of that, DURABILITY Qos needs to be VOLATILE to address only DomainParticipants that are currently in the system. {\em ParticipantStatelessMessage} is not suitable because it's not reliable nor secured. {\em ParticipantMessageSecure} Topic is not suitable because it's QoS has DURABILITY kind TRANSIENT\_LOCAL rather than VOLATILE (which is required). So new Topic {\em ParticipantVolatileMessageSecure} with associated {\em BuiltinParticipantVolatileMessageSecureWriter} and {\em BuitinParticipantVolatileMessageSecureReader} is introduced. 

\sec Plugin architecture

There are five \glref{SPI}s:
\begitems
* Authentication
* Access-Control
* Cryptographic
* Logging
* Data Tagging
\enditems

Interactions of plugins are shown in figure \ref[pic:sec-pa].

\midinsert \clabel[pic:sec-pa]{Plugin Architecture Model}
\picw=14cm \cinspic pic-sec/plugin-architecture.png
\caption/f Plugin Architecture Model (chapter 8.1 in \cite[OMG:DDS-SECURITY]).
\endinsert

\secc Authentication plugin

{\em Authentication} is process of verifying that (in this case) DomainParticipant is really the one it claims to be. DomainParticipant is authenticated when joining a \glref{DDS} Domain, mutual authentication is supported and shared secret is established between DomainPariticpants.

\secc Access Control plugin

Ensures authorization - allows or deny protectected operations of DomainPariticipant as join domain, create Topic, publish to Topic or subscribe Topic.

\secc Cryptographic plugin

Encryption, decryption, digests, \glref{MAC}, \glref{HMAC}, key generating and exchange, signing and verifying of signatures is ensured by {\em Cryptographic plugin}. The plugin \glref{API} has to be general enough to allow specific requirements for cryptographic libraries, encryption and digest algorithms, message authentication and signing users of \glref{DDS} may need to deploy.

\secc Logging plugin

This plugin logs security events of DomainParticipant. Two options of collecting log data are logging all events to a local file and distributing log events securely over \glref{DDS}.

\secc Data Tagging plugin

Classification of data is performed by {\em Data Tagging plugin}. It can be used for access control based on tag, message prioritization or even don't have to be used by middleware (\glref{RTPS} implementation), but by application or service. There are four kinds of tagging:
\begitems
* Data Writer - used in specification, data received from DataWriter has it's tag.
* Data Instance - each instance of the data has a tag.
* Individual sample - each sample of data instance is tagged individually.
* Per field - the most complex method of tagging.
\enditems

\sec Interoperability

Out-of-the-box interoperability of \glref{DDS} Security implementations is ensured analogously to \glref{RTPS} implementations - while mandatory {\em builtin endpoints} ensures that each DomainParticipant is able to discover other DomainParticipants, in \glref{DDS} Security implementations, each DomainParticipant is able to secure data by at least mandatory {\em builtin plugins}.

\secc Requirements

This is resume of requirements for builtin plugins by out-of-the-box interoperability as presented in chapter 9.2 of \cite[OMG:DDS-SECURITY]. Following are essential functional requirements for builtin plugins:
\begitems
* Authentication of DomainParticipants joining a domain
* Access control of applications subscribing to data
* Message integrity and authentication
* Encryption of a data by different keys
\enditems

Following are functions that should be required by builtin plugins:
\begitems
* Sending digitally signed data
* Sending data securely over multicast
* Data tagging
* Integrating with open standard security plugins
\enditems

Following are functions considered useful:
\begitems
* Access control to certain samples
* Access control to certain attributes within sample
* Permissions for QoS usage by \glref{DDS} Entities
\enditems

Non-functional requirements are:
\begitems
* Performance and Scalability
* Robustness and Availability
* Fit to \glref{DDS} Data-Centric Information Model
* Reuse of existing security infrastructure and technologies
* Ease of use
\enditems

\secc Considerations

Usually \glref{DDS} is deployed in systems where high performance for large number of DomainParticipants is needed, therefore actions performed by plugins shouldn't notably degrade system performance. In practice it means that asymmetric cryptography should be used only for discovery, authentication, session and shared-secret establishment, symmetric cryptography shoud be used for data, use of ciphers, \glref{HMAC}s or digital signatures should be selectable per Topic, there should be possibility of providing integrity via \glref{HMAC} without data encryption and there should be support for encrypted data over multicast.

\glref{DDS} used to be deployad in system where {\em robustness and availability} is considered critical. It's required from system to continue operating even if partial fail occures, so centralized services reprezenting single point of failure have to be avoided, DomainParticipant components have to be self-contained to be able to operate securely, multi-party key agreement protocols should be avoided because of simplicity of discruption and tokens and keys should be compartmentalized as much as possible to avoid situations where multiple applications using same key are compromised if just one of them is compromised.

\sec Implementation

The implementation of \glref{DDS} Security into \glref{ORTE} consists of changes in Modules (presented in chapter 8 of \cite[OMG:DDSI-RTPS22]). Introduction of {\em SecureSubMsg} Submessage and {\em SecuredPayload} Submessage Element assumes modification of Message Module, Discovery Module is affected by {\em Builtin Secure Endpoints} - if configured to, discovery of {\em publishers} and {\em subscribers} is secured and Behavior Module needs to be modified to include {\em Builtin Plugins} which ensures security.

\secc Builtin Endpoints

This is the list of {\em Builtin Endpoints} presented in chapter 7.4.5 of \cite[OMG:DDS-SECURITY]. In order to ensure out-of-the-box compatibility, following {\em Builtin Secure Endpoints} needs to be implemented:
\begitems
* SEDPbuiltinPublicationsSecureWriter
* SEDPbuiltinPublicationsSecureReader
* SEDPbuiltinSubscriptionsSecureWriter
* SEDPbuiltinSubscriptionsSecureReader
* BuiltinParticipantMessageSecureWriter
* BuiltinParticipantMessageSecureReader
* BuiltinParticipantVolatileMessageSecureWriter
* BuiltinParticipantVolatileMessageSecureReader
\enditems

\secc Builtin Plugins

This is resume of {\em Builtin Plugins} presented in chapter 9.1 of \cite[OMG:DDS-SECURITY]. In order to ensure out-of-the-box compatibility, following {\em Builtin Plugins} needs to be implemented:
\begitems
* \glref{DDS}:Auth:\glref{PKI}-\glref{RSA}/\glref{DSA}-\glref{DH} (Authentication plugin)
  \begitems\style x
  * Uses \glref{PKI} with pre-configured shared Certificate Authority
  * \glref{RSA} or \glref{DSA} and Diffie-Hellman for authentication and key exchange
  \enditems
* \glref{DDS}:Access:\glref{PKI}-Signed-\glref{XML}-Permissions (Access control plugin)
  \begitems\style x
  * Permissions document signed by shared Certificate Authority
  \enditems
* \glref{DDS}:Crypto:\glref{AES}-\glref{CTR}-\glref{HMAC}-\glref{RSA}/\glref{DSA}-\glref{DH} (Cryptographic plugin)
  \begitems\style x
  * \glref{AES}128 for encryption (counter mode)
  * \glref{SHA}1 and \glref{SHA}256 for digest
  * \glref{HMAC}-\glref{SHA}1 and \glref{HMAC}-256 for \glref{HMAC}
  \enditems
* \glref{DDS}:Tagging:\glref{DDS}\_Discovery (Data Tagging plugin)
  \begitems\style x
  * Send Tags via Endpoint Discovery
  \enditems
* \glref{DDS}:Logging:\glref{DDS}\_LogTopic (Logging plugin)
  \begitems\style x
  * Logs security events to a dedicated \glref{DDS} Log Topic
  \enditems
\enditems

