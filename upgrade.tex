\label[upgrade]\chap Required changes in ORTE

In this chapter, changes required for \glref{ORTE} to be compatible with the version 2.2 of the \glref{RTPS} protocol are discussed. Following is the state of work, where what is \Blue done is blue\Black, \Green working is green\Black, \Red not finished is red \Black and needless for interoperability is black.

\begitems
\Black
* Structure Module
  \begitems\style x
  \Blue
  * Participant
  * History Cache
  * Cache Change
  * Participant Proxy
  \Green
  * Writer Endpoint
  * Reader Endpoint
  * Reader Proxy
  * Writer Proxy
  \enditems

\Black
* Messages Module
  \begitems\style x
  \Blue
  * Header
  * Message Receiver
  * Data
  * InfoDestination
  * InfoReply
  * InfoSource
  * InfoTimestamp
  * Pad
  \Green
  * AckNack
  * Gap
  * Heartbeat
  \Black
  * DataFrag
  * HeartbeatFrag
  * NackFrag
  \enditems

\Black
* Behavior Module
  \begitems\style x
  \Blue
  * Best-Effort Stateless Writer
  * Best-Effort Stateless Reader
  \Red
  * Reliable Stateless Writer
  * Reliable Stateful Reader, Writer Proxy
  \Black
  * Best-Effort Stateful Writer
  * Reliable Stateful Writer, Reader Proxy
  * Best-Effort Stateful Reader
  \enditems

\Black
* Discovery Module
  \begitems\style x
  \Blue
  * SPDP
  \Red
  * SEDP
  \enditems
\enditems

\Black

In section \ref[orte:spec], specific types used in \glref{ORTE} are introduced, their purpose and legacy of previous implementation. In \ref[orte:struct], \ref[orte:message], \ref[orte:behavior] and \ref[orte:discovery], implementation of each module from chapter \ref[compare] is discussed. Proposal of consequential development is introduced in section \ref[orte:next].

\label[orte:spec]\sec ORTE specific

\secc Version 1.0

In \glref{RTPS} 1.0 implementation, the core structure is "ORTEDomain". Following are specific types contained in "ORTEDomain" structure:
\begitems
* "TaskProp"
* "TypeEntry"
* "ObjectEntry"
* "PSEntry"
* "CSTPublications"
* "CSTSubscriptions"
\enditems

"TaskProp" maintains properties of {\em Tasks}, including own socket, thread and "MessageBuffer" (used for sending and receiving data) for each {\em Task}. There are five {\em Tasks}: "taskRecvUnicastMetatraffic", "taskRecvMulticastMetatraffic", "taskRecvUnicastUserdata", "taskRecvMulticastUserdata" and "taskSend".

"TypeEntry" is database of {\em Types} used for data encapsulation, containing name of {\em Type} and functions to serialize and deserialize it.

The database of all ``endpoints'' is stored in "ObjectEntry". In "ORTEDomain", variable "objectEntry" of type "ObjectEntry" is the root of 3-layered AVL tree (\cite[www:ulan]), where each layer correspond to {\em Host Id}, {\em Application Id} and {\em Object Id}. Also, {\em Application Id} layer serves as the root for {\em Hierarchical Timer} (\cite[www:ulan]) used for timing in \glref{ORTE}.

"PSEntry", "CSTPublications" and "CSTSubscriptions" are databases of {\em Publications} and {\em Subscriptions}. In context of version 1.0 of the \glref{RTPS} protocol, ``endpoints'' used for user data communication are stored here.

Builtin ``endpoints'' are defined directly in "ORTEDomain" structure, there is nine of {\em CSTWriters} and {\em CSTReaders} used for \glref{CST} protocol:
\begitems
* "writerApplicationSelf"
* "readerManagers"
* "readerApplications"
* "writerManagers"
* "writerApplications"
* "writerPublications"
* "readerPublications"
* "writerSubscriptions"
* "readerSubscriptions".
\enditems

\secc Version 2.2

Domain is abstract term involving communication of nodes that have something in common. The core structure in implementation of version 2.2 of \glref{RTPS} protocol was therefore renamed to "ORTEDomainParticipant". Following types persisted:
\begitems
* "TaskProp"
* "TypeEntry"
* "ObjectEntry"
\enditems

"TaskProp" has the same purpose as in previous implementation, just some of names changed to be more appropriate:
\begitems\style x
* "taskRecvUnicastDiscoveryTraffic"
* "taskRecvMulticastDiscoveryTraffic"
* "taskRecvUnicastUserTraffic"
* "taskRecvMulticastUserTraffic".
\enditems

"TypeEntry" remains completely same.

Because of substitution of {\em Host Id} and {\em Application Id} for {\em Guid Prefix}, "ObjectEntry" changed from 3-layered to 2-layered AVL tree (\cite[www:ulan]), where the first layer corresponds to {\em Guid Prefix} and the second one to {\em Entity Id}. The root for {\em Hierarchical Timer} (\cite[www:ulan]) is at the {\em Guid Prefix} layer.

"ORTEEndpoint" structure was added to "ObjectEntry" at the {\em Entity Id} layer. "ORTEEndpoint" can be {\em Stateless Writer} or {\em Stateless Reader} at present (see \ref[upgrade]) and {\em Stateful Writer}, {\em Stateful Reader}, {\em Participant Proxy}, {\em Reader Proxy} and {\em Writer Proxy} will be added in the future. So one database containing all {\em Endpoints} in {\em Domain} is used for each {\em Participant} and there is no need for directly defined {\em Endpoints} or separate database of publishers and subscribers.

\label[orte:struct]\sec Structure Module

In the correspondence to \glref{PIM} (\ref[struct]) and as mentioned above, "ORTEDomain" structure changed to "ORTEDomainParticipant". Identifiers of "ORTEDomainParticipant" are "domainId", "participantId" and "guid", where "guid" is generated as discussed in \cite[www:github].

Because {\em CSTWriter} evolves to {\em Stateful Writer} and {\em CSTReader} to {\em Stateful Reader}, new structures "StatelessWriter" and "StatelessReader" are introduced in correspondence to reference implementation.

{\em Cache Change} is replacement of {\em CSChange} and is used as ``transfer unit'' for all exchanges in \glref{RTPS} 2.2.

The implementation of {\em History Cache} remains the same - it's implemented as {\em List} (\cite[www:ulan]). This manner allows to save memory by maintaining only one {\em Cache Change}, pointed from more structures as multiple matched {\em Reader Proxy} or {\em Writer Proxy}.

While the name changed for {\em CSTRemote Reader} and {\em CSTRemoteWriter} to {\em Reader Proxy} respective {\em Writer Proxy}, the function of this {\em Endpoints} remains the same.

\label[orte:message]\sec Messages Module

\label[orte:behavior]\sec Behavior Module

\label[orte:discovery]\sec Discovery Module

\label[orte:next]\sec Next Steps



