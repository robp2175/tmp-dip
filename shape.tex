\label[shape]\chap Shape for Android

\sec Shape demo

With \glref{ORTE} implementation of \glref{RTPS} 1.0 protocol, demo application called Shape is delivered. Shape demo demonstrates the functionality of \glref{ORTE} - when the color (Blue, Green, Red, Black, Yellow) is choosen, the {\em Publisher} is created as random shape (Circle, Square, Triangle) moving on the screen. Then, under the topic of color name, object's shape, color and coordinates are published to the network. It's possible to receive and interpret object's data (to see colored shapes moving on the screen) by adding the {\em Subscribers} of specific topics (colors).

\midinsert
\picw=14cm \cinspic pic-shape/qt_pub.jpg
\caption/f Shape demo - {\em Publishers} and {\em Subscribers}
\endinsert

\sec Familiarization with ORTE

The familiarization with \glref{ORTE} was done by creating demo application for Android compatible with Shape. Because the port of \glref{ORTE} to Android has been already done in \cite[FEE:vajnar-bc] and is available as library, the main task was application design and compatibility ensurance. The application was designed to be as simple as possible. {\em Publishers} view allows to create new {\em Publisher} of specific color and random shape, {\em Subscribers} view allows to set up {\em Subscribers} of specified colors. Finally, Settings and Help views are present.

\sec Classes

As in Shape demo, in Shape for Android the {\em BoxType} class is presented, allowing to create, send and receive objects. {\em BoxType} consists of {\em color} (integer), {\em shape} (integer) and {\em rectangle} ({\em BoxRect}), where {\em BoxRect} is class for storing coordinates - {\em top\_left\_x} (short), {\em top\_left\_y} (short), {\em bottom\_right\_x} (short), {\em bottom\_right\_y} (short). The {\em BoxType} is extension of {\em MessageData} class delivered with \glref{ORTE} library for Android. It allows to send and receive objects.

{\em PublisherShape} class stores {\em BoxType} information about {\em Publisher}, it's properties needed for \glref{ORTE}, methods for communication with object and prepares data to send. In {\em Publisher} view, {\em Publisher} objects are created, stored in {\em ArrayList} and drawed on screen. Data objects are sent in {\em Publisher} activity each time objects are redrawn.

{\em SubscriberElement} class receives {\em BoxType} object from \glref{ORTE} and stores it's data and methods needed for presentation. In {\em Subscriber} view, all received objects are stored in {\em ArrayList} and periodically redrawn.

Settings view allows to set up scaling, needed because of various dimensions of screens. It also contains a list of managers - in \glref{RTPS} 1.0 special application called manager is used for communication of available {\em Publishers}/{\em Subscribers} between nodes. In \glref{RTPS} 2.2 Simple Participant Discovery Protocol (\glref{SPDP}) and Simple Endpoint Discovery Protocol (\glref{SEDP}) are used.

Help view contains information about \glref{ORTE}, Shape and application usage.

\sec Compatibility

{\em BoxType} in Shape and Shape for Android is a little bit different. The reason is just familiarization with \glref{ORTE} implementation and \glref{RTPS} protocol, where misunderstooding was not fully avoided. Suggestions for improvements follows.

The first property of {\em BoxType} is {\em color}. In Shape demo, {\em color} is typed as CORBA\_octet (macro for uint8\_t, 1 byte) and in Shape for Android, {\em color} is of integer type (4 bytes). The reason why this approach does not break the compatibility is following: each data-type serialized by \glref{CORBA} is aligned to 4 byte boundary. In this case, object color is first byte and the rest until the boundary is filled by zero bytes. This data representation corresponds to Little Endian in which the message is encoded by default (endianness is operating system dependent), so when Shape for Android deserialize data, Little Endian encoded integer is obtained. It also works in opposite direction - value of the {\em color} is serialized as integer, encoded as Little Endian and on the side of Shape demo, CORBA\_octet is deserialized and 3 zero bytes skipped because of boundary alignment. The problem could arise when {\em color} would be sent as integer with Big Endian encoding and received as CORBA\_octet, because the value of the first byte would be then zero. Also, the problem wouldn't persist in the opposite direction, because endianness is always part of the \glref{RTPS} message so even node with Big Endian default encoding would receive Little Endian encoded message correctly.

The second property of {\em BoxType} is {\em shape}. The type in Shape demo is CORBA\_long (macro for int32\_t, 4 bytes) and integer (4 bytes) in Shape for Android. Therefore there is no problem with {\em shape} property.

The last property of {\em BoxType} is {\em BoxRect} consisting of coordinates of object. Each value of {\em BoxRect} is CORBA\_short type (2 bytes) in Shape demo and short type (2 bytes) in Shape for Android. Because {\em BoxRect} is presented as \glref{CORBA} autonomous data-type, the whole data-type (8 bytes) is aligned to 4 bytes boundary.

The suggestion for the future improvement of Shape demo and Shape for Android is the revision of {\em BoxType} data-type.

\midinsert
\picw=14cm \cinspic pic-shape/android_pub.jpg
\caption/f Shape for Android - Publishers view
\endinsert

