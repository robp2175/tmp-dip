\label[concl]\chap Conclusion

The aim of this thesis was upgrading Open Real-Time Ethernet implementation of Real-Time Publish-Subscribe protocol in order to achieve compatibility with the current protocol version. Actual protocol version was introduced and compared to implemented version. Required changes were discussed, implementation of basic types, messages and behavior was introduced and future development was outlined. Test of participant discovery was successfully performed even though the implementation isn't complete. Demonstration application and possibilities of security in implementation was presented.

The main difference between the originally implemented version and the current version of the protocol is that traffic used for internal purposes of protocol (e.g. discovery traffic) differs from user data traffic just in identification of endpoints. There is also difference in behavior of endpoints, namely in storage of remote endpoints information, because needs for best effort communication differs from needs for reliable one. The last important difference is related to data transfer. Because discovery and user data traffic is practically the same, only one type of message is used.

As the result, basic types were upgraded and added, messages changed and new kind of behavior was implemented. However the second type of behavior is still missing and discovery is implemented partially.

The testing of participant discovery was performed against OpenDDS implementation and remote participant with its builtin endpoints was successfully discovered. \glref{ORTE} against itself test confirmed availability of \glref{IP} address and port for multiple participants in the same domain on the same node and that generated messages are accepted by Wireshark network analyzer.

