\input ctustyle
\input opmac-bib
\worktype [O/CZ]
\workname {Návod na nastavení vývojového prostředí pro AVS kit v OS Debian}
\faculty {F3}
\department {Katedra měření}
\title {AVS Kit pro Debian}
\author {Jiří Hubáček}
\date {Březen 2015}
\makefront




\chap Úvod

AVS kit~\cite[avs-kit] s procesorem STM32F100RB, který je na bázi STM32VL DISCOVERY kitu, je používán v předmětu \uv{Aplikace vestavných systémů} pro seznámení studentů s moderními procesory s jádrem ARM cortex M3. Tento text slouží jako návod pro nastavení vývojového prostředí pro AVS kit v OS Debian. Text je distribuován pod licencí cc-by-sa~\cite[cc-by-sa].


\sec Operační systém

Návod je pro operační systém Debian GNU/Linux. Kapitola \ref[deb7] popisuje instalaci pro Debian 7.8 (wheezy), kapitola \ref[deb8] pak instalaci pro Debian 8 (jessie).




\chap Instalace nástrojů

Je třeba nainstalovat následující základní balíčky, příkazy je nutné provádět s oprávněním uživatele {\em root}.

\begtt
#apt-get install build-essential git autoconf libusb-1.0-0-dev pkg-config
\endtt


\sec ST-Link

Pro nahrávání programů do procesoru je třeba program ST-Link~\cite[st-link-github], který poběží jako daemon a bude zprostředkovávat komunikaci mezi procesorem a GDB~\fnote{GNU debugger} pro ARM. ST-Link je třeba nakopírovat do vhodného umístění - lze předpokládat, že se jedná o adresář {\bf dev} v domovském adresáři uživatele - a ručně zkompilovat.

\begtt
#cd ~/dev/
#git clone https://github.com/texane/stlink
#cd stlink/
#./autogen.sh
#./configure
#make
\endtt

\secc udev rules

Před připojením AVS kitu je třeba přidat {\em udev rules}, restartovat {\em udev} daemona, přidat modprobe konfiguraci a znovunačíst {\em usb-storage} modul. Následující příkazy je nutné provádět s oprávněním uživatele {\em root} z adresáře, kde je umístěný ST-LINK (v tomto případě {\bf dev/stlink/}).

\begtt
#cp 49* /etc/udev/rules.d/
#service udev restart
#cp stlink_v1.modprobe.conf /etc/modprobe.d/
#modprobe -r usb-storage && modprobe usb-storage
\endtt


\sec Upozornění

Před přidáním testovacího repozitáře je nutné zkontrolovat, že až doposud vše funguje. Při instalaci balíčku z testovacího repozitáře se případné chybějící balíčky stáhnou a nainstalují, ovšem také z testovacího repozitáře, což je nežádoucí. 


Následující příkaz spouští ST-Link daemona z adresáře {\bf dev/stlink/}, je nutné jej spustit až po připojení AVS kitu.

\begtt
#./st-util -1 -m
\endtt

Významy parametrů jsou {\bf -1} pro verzi 1 (STM32F100) a {\bf -m} proto, aby po odpojení GDB daemon nespadl.




\chap Instalace GDB

\label[deb7] \sec Debian 7.8 - Testing repository 

Potřebný balíček \uv{gcc-arm-none-eabi} je dostupný pouze z repozitáře {\em testing}~\cite[gcc-arm-none-eabi], je tedy třeba přidat repozitář a nastavit pining. Je důrazně varováno před experimentováním s testing repozitáři na produkčních zařízeních. Následující nastavení předpokládá instalaci balíčků z {\em testing} pouze na vyžádání. Následující příkazy je nutné provádět s oprávněním uživatele {\em root}.

\begtt
#cat << EOF >> /etc/apt/sources.list.d/testing.list
deb http://ftp.cz.debian.org/debian/ testing main contrib non-free
deb-src http://ftp.cz.debian.org/debian/ testing main
EOF
\endtt

\begtt
#cat << EOF >> /etc/apt/preferences.d/testing.pref
Package: *
Pin: release a=testing
Pin-Priority: -1
EOF
\endtt

\begtt
#apt-get update
\endtt

\begtt
#apt-get -t testing install gdb-arm-none-eabi
\endtt


\label[deb8] \sec Debian 8 - Main repository

V OS Debian 8 je potřebný balíček \uv{gcc-arm-none-eabi} již v hlavním repozitáři, není proto potřeba instalovat z testing. Následující příkazy je nutné provádět s oprávněním uživatele {\em root}.

\begtt
#apt-get install gdb-arm-none-eabi
\endtt

\sec Poznámka

S balíčkem \uv{gdb-arm-none-eabi} je nainstalován i balíček \uv{gcc-arm-none-eabi} a tím je vývojové prostředí připraveno.




\chap Práce v prostředí


\sec Spuštění ST-Link

\begtt
#./st-util -1 -m
\endtt


\sec Spuštění GDB

GDB se spouští ideálně z adresáře, kde je/bude vygenerovaný {\bf .elf} soubor. 

\begtt
#arm-none-eabi-gdb
(gdb) target remote :4242
(gdb) load main.elf
(gdb) c
\endtt

Příkaz \uv{target remote :4242} zajišťuje připojení GDB k ST-Link daemonovi, příkaz \uv{load} nahraje program {\bf main.elf} do procesoru a příkazem \uv{c} se pokračuje v běhu programu (continue).


\sec Šablona

Pro vlastní projekty je vhodné použít šablonu pro procesor STM32F100 s předem připravenými knihovnami a skripty. Šablonu lze nakopírovat do vhodného umístění pomocí gitu. Lze předpokládat, že vhodné umístění je adresář {\bf dev} v domovském adresáři uživatele.

\begtt
#cd ~/dev/
#git clone https://github.com/jeremyherbert/stm32vl-discovery-template-project
#cd stm32vl-discovery-template-project
\endtt

Soubory {\bf .c} umístit do {\bf src}, sobory {\bf .h} do {\bf inc}. V souboru {\bf Makefile} editovat řádek začínající \uv{OBJS=...}, kde je třeba mít soubory projektu s koncovkou {\bf .o}. Následně už jen spustit příkaz \uv{make} v adresáři projektu obsahujícím {\bf Makefile}.

\begtt
#make
\endtt


\sec Upozornění

Pokud projekt využívá zápis do paměti FLASH, je nutné upravit {\em ldscript}, tj. soubor {\bf stm32f100.ld}, řádek 23

\begtt
/* puvodni hodnota */
FLASH (rx) : ORIGIN = 0x08000000, LENGTH = 128K

/* hodnota nutna pro povoleni zapisu do pameti FLASH */
FLASH (rwx) : ORIGIN = 0x08000000, LENGTH = 128K
\endtt


\sec Hello World

Šablona z minulé sekce obsahuje soubory programu typu \uv{Hello World}. Pro jejich otestování stačí spustit příkaz \uv{make} z adresáře, kde je přítomen {\bf Makefile} (v tomto případě {\bf dev/stm32vl-discovery-template-project}).


\sec Poznámka

Příkaz \uv{make} je možné spouštět i z konzole GDB pro ARM, což ušetří jeden panel při vývoji.




\chap Závěr

V tuto chvíli by mělo být nainstalováno funkční prostředí pro vývoj na AVS kitu a otestován program typu \uv{Hello World}. Za největší přínos je považována šablona prázdného projektu obsahující {\bf Makefile} s nastavením závislostí knihoven a ldskriptem pro procesor.


\sec Úpravy dokumentu

\begitems
* 21. 6. 2015 - přidán postup instalace na Debian 8
* 8. 4. 2015 - doplnění upozornění o zapisování do paměti flash
* 20. 3. 2015 - úvodní verze návodu (instalace nástrojů, instalace z testing repository, práce v prostředí)
\enditems


\bibchap
\usebib/c (simple) mybase

\bye

