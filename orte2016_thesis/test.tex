\label[test]\chap Testing of implementation

In reference to \ref[upgrade], only \glref{SPDP} tests were performed. Interoperability was tested against OpenDDS implementation \cite[www:opendds] - open source implementation of \glref{OMG} \glref{DDS} delivered with \glref{RTPS} Discovery test and it was run between 64-bit Debian \glref{PC} and 32-bit Linux Mint \glref{PC}. The main purpose of \glref{ORTE} against itself test was {\em Default Multicast Locator}'s \glref{IP} address and port availability for more {\em Participants} in same {\em Domain} on same node. Test was performed on 64-bit Debian \glref{PC}.

\sec ORTE to OpenDDS

The following is commented ``debug`` output for typical \glref{RTPS} {\em Messsage} containing {\em SPDPdiscoveredParticipantData}, encoded as {\em Parameter List}.

\begtt
msg len 236
header OK
pv 2.2 vid 1.3 gp 0x0103001e 0x33862b64 0x76c10000
\endtt
RTPS Header received successfully.

\begtt
 15 05 00 00  ; 15 = DATA; 05 = D,LE; 00 00 = till the end
 00 00 10 00  ; 00 00 = extra flags; 10 00 = 16B to next header
 00 00 00 00  ; Reader Id = ENTITYID_UNKNOW
 00 01 00 c2  ; Writer Id = ENTITYID_SPDP_BUILTIN_PARTICIPANT_WRITER
 00 00 00 00  ; SequenceNumber.high = 0
 01 00 00 00  ; SequenceNumber.low  = 1
\endtt
RTPS Data Submessage Header received successfully.

\begtt
 00 03 00 00  ; 00 03 = PL_CDR_LE; 00 00 = skipped
\endtt
Information in {\em Data Submessage} is encoded as {\em Parameter List}, both the {\em Parameter List} and its {\em Parameters} are encapsulated using Little Endian.

\begtt
 15 00 04 00  ; 15 00 = Protocol Version; 04 00 = length 4B
 02 02 00 00  ; 2.2
\endtt
The protocol version of sending implementation is 2.2.

\begtt
 50 00 10 00  ; 50 00 = Participant GUID; 10 00 = length 16B
 01 03 00 1e  ; GuidPrefix
 33 86 2b 64  ; GuidPrefix
 76 c1 00 00  ; GuidPrefix
 00 00 01 c1  ; Entity Id = ENTITYID_PARTICIPANT
\endtt
The \glref{GUID} of the sending {\em Participant} is received.

\begtt
 16 00 04 00  ; 16 00 = Vendor Id; 04 00 = length 4B
 01 03 00 00  ; 1.3
\endtt
The vendor id of sending implementation is 1.3.

\begtt
 44 00 04 00  ; 44 00 = Builtin Endpoints; 04 00 = length 4B
 3f 0c 00 00  ; All required Builtin Endpoints present
 58 00 04 00  ; 58 00 = Builtin Endpoint Set; 04 00 = length 4B
 3f 0c 00 00  ; All required Builtin Endpoints present
\endtt
Sending {\em Participant} contains all {\em Builtin Endpoints} required for interoperability.

\begtt
 32 00 18 00  ; 00 32 = Metatraffic Unicast Locator; 18 00 = length 24B
 01 00 00 00  ; Locator.kind = LOCATOR_KIND_UDPv4
 7f a9 00 00  ; Locator.port = 43391
 00 00 00 00  ;
 00 00 00 00  ;
 00 00 00 00  ;
 c0 a8 01 75  ; Locator.address = 192.168.1.117
\endtt
Metatraffic Unicast Locator\fnote{Attributes of "Locator" structure introduced in \ref[orte:struct] are "kind", "port" - both integers and "address" - the array of 16B. As mentioned at start of {\em Parameter List}, Little Endian encapsulation is used in {\em Data Submessage}. Therefore {\em Locator}s kind and port are Little Endian encoded but address is sent per byte in network order.} is used for communicating {\em Discovery Traffic}. It contains \glref{IP} address and port on which are sending {\em Participants} {\em Builtin Endpoints} listening.

\begtt
 32 00 18 00  ; 00 32 = Metatraffic Unicast Locator; length 24B
 01 00 00 00  ; Locator.kind = LOCATOR_KIND_UDPv4
 7f a9 00 00  ; Locator.port = 43391
 00 00 00 00  ;
 00 00 00 00  ;
 00 00 00 00  ;
 0a 01 02 04  ; Locator.address = 10.1.2.4
\endtt
The second Metatraffic Unicast Locator is received, meaning that sending {\em Participants} {\em Builtin Endpoints} are listening on multiple \glref{IP} addresses (and same ports in this case).

\begtt
 31 00 18 00  ; 31 00 = Default Unicast Locator; 18 00 = length 24B
 01 00 00 00  ; Locator.kind = LOCATOR_KIND_UDPv4
 39 30 00 00  ; Locator.port = 12345
 00 00 00 00  ;
 00 00 00 00  ;
 00 00 00 00  ;
 7f 00 00 01  ; Locator.address = 127.0.0.1
\endtt
{\em Default Unicast Locator} is used when \glref{IP} address and port aren't known for destination {\em User Endpoint}. In that case, {\em Default Unicast Locator} is used for sending {\em Unicast User Data Traffic}.

\begtt
 48 00 18 00  ; 48 00 = Default Multicast Locator; 18 00 = length 24B
 01 00 00 00  ; Locator.kind = LOCATOR_KIND_UDPv4
 39 30 00 00  ; Locator.port = 12345
 00 00 00 00  ;
 00 00 00 00  ;
 00 00 00 00  ;
 7f 00 00 01  ; Locator.address = 127.0.0.1
\endtt
As well as {\em Default Unicast Locator}, also {\em Default Multicast Locator} is mandatory attribute of {\em Participant}. It has the same meaning, except {\em Multicast User Data Traffic} is considered. 

\begtt
 34 00 04 00  ; 34 00 = Manual Liveliness Count; 04 00 = length 4B
 00 00 00 00  ;
\endtt
{\em Manual Liveliness Count} parameter is used for {\em Writer Liveliness Protocol} with manual Qos settings.

\begtt
 02 00 08 00  ; 02 00 = Participant Lease Duration; 08 00 = length 8B
 14 00 00 00  ; NtpTime.seconds = 20
 00 00 00 00  ; NtpTime.fraction = 0
\endtt
Parameter containing the time period after which the sending {\em Participant} should be removed from "objectEntry" database.

\begtt
 01 00 00 00  ; 01 00 = Sentinel (end of Parameter List)
\endtt
Sentinel determines the end of {\em Parameter List}.

The following is debug output of all {\em Entities} stored in "objectEntry" database of {\em Participant}. The whole output of receiving and processing \glref{RTPS} {\em Message} with \glref{SPDP} traffic is presented in appendix \ref[app:test1].

\begtt
1450639460.456 | GP:   00000000 0x0000c0a8 0x016525bc
1450639460.457 |   EID: 0x00000104
1450639460.457 |   EID: 0x000001c1
1450639460.457 |   EID: 0x000100c2
1450639460.457 |   EID: 0x000100c7
1450639460.457 | GP: 0x0103001e 0x33862b64 0x2b700000
1450639460.457 |   EID: 0x000001c1
1450639460.457 |   EID: 0x000003c2
1450639460.457 |   EID: 0x000003c7
1450639460.457 |   EID: 0x000004c2
1450639460.457 |   EID: 0x000004c7
1450639460.457 |   EID: 0x000100c2
1450639460.457 |   EID: 0x000100c7
1450639460.457 |   EID: 0x000200c2
1450639460.457 |   EID: 0x000200c7
\endtt

Because the "objectEntry" database contains remote {\em Participant} and it's builtin {\em Endpoints}, test of the \glref{SPDP} was successful.

\sec ORTE to ORTE

It was successfully tested that {\em Default Multicast Locator}'s \glref{IP} address and port can be shared. Two {\em Participants} with different "participantId" and with the same "domainId" was run on 64-bit Debian \glref{PC}. They both fill it's "objectEntry" database with remote {\em Participant} and it's builtin {\em Endpoints} with no problems. Packet captured by Wireshark\fnote{Network protocol analyzer} containing \glref{RTPS} {\em Message} with \glref{SPDP} traffic is presented in appendix \ref[app:test2].

