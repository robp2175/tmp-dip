\label[future]\chap Future Development

\glref{ORTE} is written to fit \glref{RTPS} 1.0. In version 2.2 there is no difference between meta traffic and user traffic, specification is divided to modules and the fashion of {\em Participants} and {\em Endpoints} discovery is revised. These are main differences that should be considered when upgrading \glref{ORTE}. In the following sections, changes that haven't been made yet are discussed.

\sec Stateful Endpoints

It's not necessary to implement {\em Stateful Reference Endpoints}, only required behavior is discussed in \ref[inte]. However, for reliable communication desired by \glref{SEDP}, at least some state on each matched {\em Writer} must be stored in {\em Reader}. Also backward compatibility with version 1.0 of the \glref{RTPS} protocol would be easier, because approach similar to {\em Stateful Reference Implementation} is used. Therefore, it's advisable to add {\em Stateful Reference Implementation} into shor-term goals.

\sec Refactoring

Terminology changed in version 2.2 and in some cases (e.g. "MessageInterpret") \glref{ORTE} implementation doesn't correspond with specification. Clean code, suitable namespace and documentation is fundamental for changes, upgrades and fixes.

\sec Directory Structure

Four modules are introduced in \cite[OMG:DDSI-RTPS22] - Structure, Messages, Behavior and Discovery. It would be easier to maintain the code if these are stored in proper directory structure. Also additional directories for documentation can be considered.

\sec DDS API

It's mentioned in \cite[ORTE:conf] that first version of \glref{DDS} specification was introduced when it wasn't possible to change \glref{ORTE} \glref{API}. If there is an opportunity, it would be great to change it in correspondence to \glref{DDS}.

\sec Backward compatibility

One of the long-term goals could be backward compatibility with \glref{RTPS} 1.0. Ideally it would be implemented similar to \glref{SPDP} - as {\em User Application} with {\em Builtin Endpoints}.

\sec Use Cases

Some use cases can be considered when the \glref{RTPS} protocol is implemented in one {\em Node} (physical device) of {\em Network}. For this purpose, {\em Interpreter Submessages} discussed in \ref[message] were introduced.

\begitems
* 1 domain, 1 participant
* 1 domain, x participants
* y domains, 1 participant
* y domains, x participants
\enditems

In the {\em 1-1} case, there are no special needs. Actual development consider this use case.

For the {\em 1-x} case, sources consumption can be reduced by sharing common parts of {\em Domain} with other {\em Participants}.

In the {\em y-1} case, {\em Participants} in different {\em Domains} listen on different ports. However sending sources can be shared.

The last case {\em y-x} would benefit from the {\em 1-x} and {\em y-1} cases.

