\label[compare]\chap Actual RTPS protocol

When the {\em User Application} needs to exchange {\em Data Object} beetween multiple nodes in the network ({\em Entities} in the {\em Domain} in terms of \glref{RTPS}), the \glref{RTPS} protocol suits perfectly.

This chapter follows Platform Independent Model (\glref{PIM}) of the \glref{RTPS} protocol introduced in \cite[OMG:DDSI-RTPS22]. \glref{PIM} is divided in four modules: basic objects are discussed in section \ref[struct], messages used for communication are described in \ref[message] and behavior - messages exchange between objects - is discussed in \ref[behavior]. The last module of discovering {\em Domain} is convered in section \ref[discovery]. In the last section \ref[rtps10], versions 2.2 and 1.0 of \glref{RTPS} protocol are compared.

%%%%%%%%%%%%%%%%%%%%%%%%%%%%%%%%%%%%%%%%%%%%%%%%%%%%%%%%%%%%%%%%%%%%%%%%%%%%%%%%
%%%%%%%%%%%%%%%%%%%%%%%%%%%%%% STRUCTURE MODULE %%%%%%%%%%%%%%%%%%%%%%%%%%%%%%%%
%%%%%%%%%%%%%%%%%%%%%%%%%%%%%%%%%%%%%%%%%%%%%%%%%%%%%%%%%%%%%%%%%%%%%%%%%%%%%%%%
\label[struct]\sec Structure Module

The communication take place in the \glref{RTPS} {\em Domain}, consisting of multiple {\em Entities}. Each {\em Entity} can be either {\em Participant} or {\em Endpoint}, where {\em Endpoints} can be specialized as {\em Writer} or {\em Reader}. Each {\em Endpoint} has it's own database of {\em Cache Changes} called {\em History Cache}. The whole structure is shown in figure \ref[pic:struct].

\midinsert \clabel[pic:struct]{Structure Module diagram}
\picw=14cm \cinspic pic-compare/structure_module.png
\caption/f Structure Module diagram (chapter 7.1 in \cite[OMG:DDSI-RTPS22]).
\endinsert

It should be mentioned that there is also proxy {\em Participant} - {\em Participant Proxy} and another two proxy {\em Endpoints} - {\em Reader Proxy} and {\em Writer Proxy}. These objects represents remote {\em Participants} and their {\em Writers} and {\em Readers} and are introduced in chapter 8.4 in \cite[OMG:DDSI-RTPS22]. The local {\em Participant} maintains the topology of the {\em Domain} and needs to store information about remote {\em Participants}. For this purpose, {\em Participant Proxy} is used. Sometimes, local {\em Writer} needs to store information about remote {\em Reader} and therefore, {\em Reader Proxy} is used. Also local {\em Readers} are sometimes in need of storing information about remote {\em Writers} and then {\em Writer Proxy} is used.

\secc Participant

{\em Domain Participant} is container for {\em Endpoints} within the same application. It has the following attributes:
\begitems
* \glref{GUID}
* Protocol Version
* Vendor Id
* Default Unicast Locator List
* Default Multicast Locator List
\enditems

Where the {\em \glref{GUID}} is globally-unique \glref{RTPS}-entity identifier consisting of {\em \glref{GUID} Prefix} and {\em EntityId}, {\em Protocol Version} is version of actual implementation and vendor of implementation is represented by {\em Vendor Id}.

{\em Default Unicast Locator List} and {\em Default Multicast Locator List} are lists of \glref{IP} address and port combinations used to send user data traffic to, when there is no such an information contained in {\em Writers} of the {\em Participant}.

Each {\em Endpoint} within the same {\em Participant} has to have the same {\em \glref{GUID} Prefix}.

\secc Writer Endpoint

{\em Writer} is the source of {\em Cache Changes} which are sent to {\em Readers}. It has the following attributes:
\begitems
* \glref{GUID}
* Topic Kind
* Reliability Level
* Unicast Locator List
* Multicast Locator List
* Push Mode
* Heartbeat Period
* Nack Response Delay
* Nack Suppression Duration
* Last Change Sequence Number
* Writer Cache
\enditems

Where {\em Topic Kind} can be either NO\_KEY or WITH\_KEY. WITH\_KEY is used, when the topic consists of more than one data instances identified by {\em key}. {\em Reliability Level} can be either BEST\_EFFORT or RELIABLE, saying if it should be verified that {\em Cache Change} reached the {\em Reader}.

{\em Unicast Locator List} and {\em Multicast Locator List} are lists of \glref{IP} address and port combinations on which is the {\em Writer} listening. If lists are empty, it's presumed that {\em Writer} is listening on {\em Default Unicast Locator List} respective {\em Default Multicast Locator List} of the {\em Participant}.

{\em Push Mode} defines if data are sent ({\em Push Mode} is set to TRUE) or Heartbeats with Sequence Numbers of available {\em Cache Changes} ({\em Push Mode} is set to FALSE). In the second case, {\em Reader} has to ask for {\em Cache Change} delivery.

{\em Heartbeat Period}, {\em Nack Response Delay} and {\em Nack Suppression Duration} are protocol tunning parameters, which defines announce interval of available data, how long the response to data request should be delayed respective how long can be request for just sent data ignored.

{\em Last Change Sequence Number} is the highest Sequence Number in {\em History Cache} and the {\em Writer Cache} is the {\em History Cache} of the {\em Writer} containing {\em Cache Changes} associated with the {\em Writer} itself.

\secc Reader Endpoint

{\em Reader} is the destination of {\em Cache Changes} which are sent by {\em Writers}. It has the following attributes:
\begitems
* \glref{GUID}
* Topic Kind
* Reliability Level
* Unicast Locator List
* Multicast Locator List
* Expects Inline Qos
* Heartbeat Response Delay
* Heartbeat Suppression Duration
* Reader Cache
\enditems

Where {\em Unicast Locator List} and {\em Multicast Locator List} are lists of \glref{IP} address and port combinations on which is the {\em Reader} listening. If there is no IP address and port combination in list, it's presumed that {\em Reader} is listening on {\em Default Unicast Locator List} respective {\em Default Multicast Locator List} of the {\em Participant}.

The value of {\em Expects Inline Qos} is set to TRUE if the {\em Reader} expects in-line Qos to be sent along with data.

{\em Heartbeat Response Delay} and {\em Heartbeat Suppression Duration} are time parameters used for protocol tunning, which defines how long the acknowledgement of data should be delayed respective how long can be Heartbeat announces ignored after just received Heartbeat.

{\em Reader Cache} is the {\em History Cache} of the {\em Reader} containing {\em Cache Changes} associted with the {\em Reader} itself.

\secc History Cache

{\em History Cache} is the database of {\em Cache Changes} serving as the \glref{API} for {\em Writer} and {\em Reader} {\em Endpoints}. It has the following attributes:
\begitems
* Changes
\enditems

Where {\em Changes} are {\em Cache Changes} stored in the {\em History Cache}.

\secc Cache Change

Is the change of the data object that should be propagated from the {\em Writer} to the matching {\em Readers}. It has the following attributes:
\begitems
* Change Kind
* Writer \glref{GUID}
* Instance Handle
* Sequence Number
* Data value
\enditems

Where {\em Change Kind}\fnote{Possible values are: ALIVE, NOT\_ALIVE\_DISPOSED and NOT\_ALIVE\_UNREGISTERED} is used to distinguish the change that was made to a data object. {\em Writer \glref{GUID}} is the identifier of the source of the {\em Cache Change}, {\em Instance Handle} identifies the instance of the data object (in \glref{DDS} the value of the {\em key} is used) and the {\em Sequence Number} is unique identifier of the {\em Cache Change} in the {\em History Cache} of the {\em Endpoint}. The last attribute, {\em Data value}, represents data associated to the {\em Cache Change}.

\secc Participant Proxy

Represents the information about remote {\em Participant} in the {\em Domain}. It has the following attributes:
\begitems
* Protocol Version
* Guid Prefix
* Vendor Id
* Expects Inline Qos
* Available Builtin Endpoints
* Metatraffic Unicast Locator List
* Metatraffic Multicast Locator List
* Default Multicast Locator List
* Default Unicast Locator List
* Manual Liveliness Count
* Lease Duration
\enditems

Where {\em Protocol Version} specify the version of the \glref{RTPS} protocol implementation used by remote {\em Participant} and the vendor of this implementation is represented by the {\em Vendor Id}. {\em Guid Prefix} is the common part of the \glref{GUID} for the {\em Participant} and all of it's {\em Endpoints}, {\em Expects Inline Qos} describes whether the {\em Readers} of the remote {\em Participant} expects in-line Qos sent along with data and {\em Available Builtin Endpoints} parameter specify which builtin {\em Endpoints} used for plug-and-play interoperability are available by remote {\em Participant}.

{\em Metatraffic Unicast Locator List} and {\em Metatraffic Multicast Locator List} are \glref{IP} address and port combinations that can be used to reach the remote builtin {\em Endpoints} and {\em Default Unicast Locator List} and {\em Default Multicast Locator List} are \glref{IP} address and port combinations that can be used to reach the remote {\em Endpoints} defined by user that serve for user data exchange.

{\em Manual Liveliness Count} is used to implement MANUAL\_BY\_PARTICIPANT liveliness Qos and {\em Lease Duration} specify the time period for which the remote {\em Participant} should be considered alive.

\secc Reader Proxy

Represents the information about remote {\em Reader}. It has the following attributes:
\begitems
* Remote Reader \glref{GUID}
* Unicast Locator List
* Muticast Locator List
* Changes for Reader
* Expects Inline Qos
* Is Active
\enditems

Where {\em Remote Reader \glref{GUID}} is unique identifier of remote {\em Reader}, {\em Unicast Locator List} and {\em Multicast Locator List} are lists of \glref{IP} address and port combinations on which is the remote {\em Reader} listening, {\em Changes for Reader} is the list of {\em Cache Changes} that should be sent to the remote {\em Reader}, {\em Expects Inline Qos} attribute specify if the remote {\em Reader} expects in-line Qos to be sent along with data and attribute {\em Is Active} is set to TRUE if the remote {\em Reader} is responsive to the local {\em Writer}.

\secc Writer Proxy

Represents the information about remote {\em Writer}. It has the following attributes:
\begitems
* Remote Writer \glref{GUID}
* Unicast Locator List
* Multicast Locator List
* Changes from Writer
\enditems

Where {\em Remote Writer \glref{GUID}} is unique identifier of remote {\em Writer}, {\em Unicat Locator List} and {\em Multicast Locator List} are lists of \glref{IP} address and port combinations on which is the remote {\em Writer} listening and {\em Changes from Writer} is the list of {\em Cache Changes} that are received or expected from the remote {\em Writer}.

%%%%%%%%%%%%%%%%%%%%%%%%%%%%%%%%%%%%%%%%%%%%%%%%%%%%%%%%%%%%%%%%%%%%%%%%%%%%%%%%
%%%%%%%%%%%%%%%%%%%%%%%%%%%%%% MESSAGES MODULE %%%%%%%%%%%%%%%%%%%%%%%%%%%%%%%%%
%%%%%%%%%%%%%%%%%%%%%%%%%%%%%%%%%%%%%%%%%%%%%%%%%%%%%%%%%%%%%%%%%%%%%%%%%%%%%%%%
\label[message]\sec Messages Module

For communication between {\em Writers} and {\em Readers}, \glref{RTPS} messages are used. Each message consists of {\em Header} and one or more {\em Submessages}, where each {\em Submessage} has it's own {\em Submessage Header} and {\em Submessage Elements} based on the kind of the {\em Submessage}. The structure of \glref{RTPS} message is shown in figure \ref[pic:msg].

\midinsert \clabel[pic:msg]{Structure of RTPS message}
\picw=14cm \cinspic pic-compare/messages_module.png
\caption/f Structure of \glref{RTPS} message (chapter 8.3.3 in \cite[OMG:DDSI-RTPS22]).
\endinsert

The interpretation of {\em Submessage} may depend on previously received {\em Submessages} within the same {\em Message}, therefore information shared between these {\em Submessages} must be stored. {\em Message Receiver} ensures this function by storing information about {\em Source Protocol Version}, {\em Source Vendor Id}, {\em Source \glref{GUID} Prefix}, {\em Destination \glref{GUID} Prefix}, {\em Unicast Reply Locator List}, {\em Multicast Reply Locator List}, {\em Have Timestamp} and {\em Timestamp}. State of the {\em Message Receiver} is reset to default values each time the {\em Message} is received.

Submesssages can be divided to {\em Entity Submessages} which target an \glref{RTPS} {\em Entity} affecting it's behavior and {\em Interpreter Submessages} changing the state of the {\em Message Receiver}.

\secc Header

The first change in the state of {\em Message Receiver} is made by {\em Header} of the received {\em Message}. The {\em Header} identify \glref{RTPS} {\em Message}, {\em Protocol Version} used, {\em Vendor} of the implementation and common part of \glref{GUID} - {\em \glref{GUID} Prefix} - used to interpret source {\em EntityId} in the {\em Submessage}.

\secc Submessage Header

{\em Submessage Header} is included in each {\em Submessage}, it has the following attributes:
\begitems
* Submessage Kind
* Flags
* Submessage Lenght
\enditems

Where {\em Submessage Kind} specify the meaning of {\em Submessage}, {\em Flags} is an array of 8 bits, which identifies endianness used in the {\em Submessage} (\glref{LSB}), the presence of optional elements and possibly changes the interpretation of the {\em Submessage}. The {\em Submessage Length} specify the length of the {\em Submessage}.

\secc Interpreter Submessages

List of {\em Submessages} and their changes to the {\em Message Receiver} follows.

{\em\bf InfoDestination} is sent from {\em Writer} to {\em Reader} to modify the {\em Destination \glref{GUID} Prefix} value of {\em Message Receiver} used to interpret {\em Reader} {\em EntityId} in the {\em Submessage}.

{\em\bf InfoReply} is sent from {\em Reader} to {\em Writer} to explicitly define where to send a reply to the {\em Submessage} that follow.

{\em\bf InfoSource} modifies the source {\em Protocol Version}, {\em Vendor Id} and {\em \glref{GUID} Prefix} of the {\em Submessages} that follow.

{\em\bf InfoTimestamp} is used to send the {\em Timestamp} that apply to the {\em Submessages} that follow.

\secc Entity Submessages

List of {\em Submessages} and their interpretation follows.

{\em\bf AckNack} is sent by {\em Reader} to inform the {\em Writer} about which sequence numbers are received and which are missing.

{\em\bf Data} is sent from {\em Writer} to {\em Reader} and is used to inform about changes to a data object. Changes may be in value or in lifecycle of the object.

{\em\bf DataFrag} is used when the {\em\bf Data} {\em Submessage} exceeds the \glref{MTU} of lower communication layers, otherwise it has the same function as {\em\bf Data} {\em Submessage}.

{\em\bf Gap} sent by {\em Writer} indicates to the {\em Reader} which sequence numbers are no longer relevant.

{\em\bf Heartbeat} announces to {\em Readers} sequence numbers of {\em Cache Changes} that are available in {\em Writer}.

{\em\bf HeartbeatFrag} is used when fragmentation occures and until all fragments are available. Once all fragments are available, {\em\bf Heartbeat} {\em Submessage} is used.

{\em\bf NackFrag} is sent by {\em Reader} to inform the {\em Writer} about which fragments are missing.

{\em\bf Pad} is padding message used for desired alignment. It has no other meaning.

%%%%%%%%%%%%%%%%%%%%%%%%%%%%%%%%%%%%%%%%%%%%%%%%%%%%%%%%%%%%%%%%%%%%%%%%%%%%%%%%
%%%%%%%%%%%%%%%%%%%%%%%%%%%%%% BEHAVIOR MODULE %%%%%%%%%%%%%%%%%%%%%%%%%%%%%%%%%
%%%%%%%%%%%%%%%%%%%%%%%%%%%%%%%%%%%%%%%%%%%%%%%%%%%%%%%%%%%%%%%%%%%%%%%%%%%%%%%%
\label[behavior]\sec Behavior Module

The purpose of the \glref{RTPS} protocol can be simplified to propagating {\em Cache Change} from {\em Writer} to {\em Reader}. The manner of propagation in relation to properties of communication is discussed in this section. There are two main properties of communication:
\begitems
* Topic Kind
* Reliability Level
\enditems

Where {\em Topic Kind} can be either NO\_KEY or WITH\_KEY and {\em Reliability Level} can be either BEST\_EFFORT or RELIABLE. Pairing of {\em Writers} with {\em Readers} must acknowledge the following restrictions.

{\em Writer} and {\em Reader} can be paired up and the communication may begin when the {\em Topic Kind} matches, because both {\em Endpoints} relate to the same \glref{DDS} Topic which is either NO\_KEY or WITH\_KEY.

The reliability depends on the {\em Reader}. If the {\em Reader} has the {\em Reliability Level} set to RELIABLE, the corresponding {\em Writer} has to also have the {\em Reliability Level} set to RELIABLE, while the {\em Reader} has the {\em Reliability Level} set to BEST\_EFFORT, the {\em Reliability Level} of the corresponding {\em Writer} doesn't matter.

The behavior required for interoperability between implementations is summarized in section \ref[inte]. In chapter 8.4 in \cite[OMG:DDSI-RTPS22], there are two reference implementations of {\em Endpoints} that are summarized in section \ref[impl]. In the remaining sections, these implementations are discussed in detail.

\label[inte]\secc Interoperability

In order to be compliant with protocol specification and interoperable with other implementations, the implementation of the \glref{RTPS} protocol must satisfy the following requirements.
\begitems
* General Requirements:
  \begitems\style x
  * Only \glref{RTPS} messages are used for communication.
  * Message Receiver must be implemented.
  * Timing characteristics must be tunable.
  * Simple Participant Discovery Protocol must be implemented.
  * Simple Endpoint Discovery Protocol must be implemented.
  * Writer Liveliness Protocol must be implemented.
  \enditems

* Required Writer Behavior:
  \begitems\style x
  * Data must be sent in-order.
  * If requested, in-line Qos must be included.
  \enditems

* Additional requirements for Reliable Writer:
  \begitems\style x
  * Periodic HEARTBEAT messages must be sent.
  * Response to ACKNACK message must be eventually sent.
  \enditems

* Required Reader Behavior:
  \begitems\style x
    * No specific behavior needed as best-effort reader is completely passive.
  \enditems

* Additional requirements for Reliable Reader:
  \begitems\style x
  * Response to HEARTBEAT with final flag not set must be eventually sent.
  * Response to HEARTBEAT indicating missing sample must be eventually sent.
  * Once acknowledged, always acknowledged.
  * ACKNACK can be only sent in response to HEARTBEAT.
  \enditems

\enditems

{\em\bf Writer Liveliness Protocol} is required by \glref{DDS} to exchange information about the Liveliness of {\em Writers} by {\em Participants}. Builtin {\em Endpoints} are used for sending samples at rate faster than the smallest lease duration among {\em Writers} sharing the same liveliness Qos.

\label[impl]\secc State Maintenance

In \cite[OMG:DDSI-RTPS22], two reference implementations of \glref{RTPS} protocol are introduced.

{\em\bf Stateless} implementation maintains no state about remote {\em Endpoints}, which suits for best-effort communication over multicast. While this implementation scales well in large systems and memory usage is reduced, additional bandwidth may be required.

{\em\bf Stateful} implementation, on the other hand maintains full state on remote {\em Endpoints}. More memory is needed and scalability is limited, but bandwidth usage decreases. Also, strict reliable communication and Qos based filtering on the {\em Writer's} side may be applied.

It's important to mention that these are {\em reference implementations}. It means that there is no need for two kinds of {\em Endpoints} - {\em Stateless} and {\em Stateful}, because the only behavior needed for interoperability is introduced in \ref[inte]. {\em Reference implementations} just helps to understand and implement this behavior. Both {\em reference implementations}, {\em Stateless} as well as {\em Stateful} may be BEST\_EFFORT, RELIABLE, NO\_KEY or WITH\_KEY. Exceptions are discussed and explained below.

\secc Stateless Writer

To be compliant with \ref[impl], {\em Stateless Writer} must send data in order and include in-line Qos if needed. BEST\_EFFORT {\em Stateless Writer} has no other requirements and data are sent each {\em Resend Data Period} or on demand by user application.

If {\em Stateless Writer} is RELIABLE, periodic HEARTBEATs with available data are sent. Sending of data then depend on {\em Push Mode}. If {\em Push Mode} is TRUE, data are sent each {\em Resend Data Period} or on demand by user application, if it's FALSE, data are stored in {\em History Cache} of {\em Writer} and sent only in response to ACKNACK.

{\em Readers} uses ACKNACK messages to request interesting data from {\em Reliable Writer}. DATA submessage is sent if data are still available in {\em Writer's} {\em History Cache} or GAP submessage indicating that data are no longer relevant.

Even the bandwidth can be reduced if the information of acknowledged Sequence Numbers would be stored for remote {\em Readers}, it's not interoperability issue.

\secc Stateless Reader

The function of BEST\_EFFORT {\em Stateless Reader} is to receive and process data. However, to ensure RELIABLE communication, the problem arise, because at least sequence numbers of announced, requested but not received {\em Cache Changes} are needed, so some information about remote {\em Writer} has to be stored and therefore {\em Reader} can't be {\em Stateless}.

As mentioned above, this is the only exception. {\em Stateless Reader} can't be RELIABLE.

\secc Stateful Writer

For each remote {\em Reader}, {\em Stateful Writer} stores information in {\em Reader Proxy} structure. When {\em Cache Change} is added to the {\em History Cache} of {\em Writer}, filtering can occures to determine if {\em Cache Change} is relevant for {\em Reader} and consequently stored in {\em Changes for Reader} of {\em Reader Proxy}.

BEST\_EFFORT traffic is sent on demand by user application to each {\em Reader Proxy} whenever there are any unsent changes in {\em Changes for Reader}.

For RELIABLE {\em Stateful Writer}, periodic HEARTBEATs must be sent to each {\em Reader Proxy}. Sending of data then depends on {\em Push Mode} - when the value is TRUE, {\em Cache Change} pass the filter and is added to {\em Changes for Reader}, the {\em Cache Change} is marked as unsent and will be sent as soon as the needed resources would be available. When tha value of {\em Push Mode} is FALSE, only HEARTBEATs are sent and {\em Reader} has to ask for data by sending ACKNACK for interested data. In response to ACKNACK, {\em Reliable Writer} then sends DATA submessage if the data are still available or GAP submessage when the data are no longer relevant.

It should be mentioned that in general, {\em Reader's Entity Id} of submessages is set to ENTITYID\_UNKNOWN, stating that each {\em Reader} of remote {\em Participant} should receive the data. The only situation when {\em Writer} knows exactly the destination and therefore may set {\em Reader's Entity Id} properly is sending DATA submessage in response to ACKNACK by {\em Stateful Writer}.

\secc Stateful Reader

For BEST\_EFFORT traffic, {\em Stateful Reader} stores information about expected Sequence Number for each remote  {\em Writer}. This information is stored in {\em Writer Proxy} structure. Storing Sequence Number ensures that there are no duplicated nor out-of-order data changes.

If the communication is RELIABLE and DATA submessage or GAP is received, expected Sequence Number is set correspondingly. When HEARTBEAT is received, databases of missing and lost changes are updated for {\em Writer Proxy} and the rest of behavior depends on {\em Final} and {\em Liveliness Flags}. When both, {\em Final} and {\em Liveliness Flags} are set, nothing happens. When {\em Liveliness Flag} is not set, then ACKNACK with missing changes may be sent and when {\em Final Flag} is not set, ACKNACK must be sent.


%%%%%%%%%%%%%%%%%%%%%%%%%%%%%%%%%%%%%%%%%%%%%%%%%%%%%%%%%%%%%%%%%%%%%%%%%%%%%%%%
%%%%%%%%%%%%%%%%%%%%%%%%%%%%%% DISCOVERY MODULE %%%%%%%%%%%%%%%%%%%%%%%%%%%%%%%%
%%%%%%%%%%%%%%%%%%%%%%%%%%%%%%%%%%%%%%%%%%%%%%%%%%%%%%%%%%%%%%%%%%%%%%%%%%%%%%%%
\label[discovery]\sec Discovery Module

The communication as described in \ref[behavior] between {\em Endpoints} described in \ref[struct] by the {\em Messages} described in \ref[message] assumes that both ends of communication are known. {\em Discovery Module} introduces two processes of probing {\em Domain} due to discovering {\em Participants} called {\em Simple Participant Discovery Protocol} and theirs {\em Endpoints} known as {\em Simple Endpoint Discovery Protocol}.

\glref{SPDP} and \glref{SEDP} are only discovery protocols described in \glref{RTPS} specification and have to be implemented in order to enable interoperability between implementations. However vendor specific discovery protocols can be implemented in addition to overcome some drawbacks of {\em Simple Discovery Protocols}.

The difference between {\em Builtin} and {\em User-defined} {\em Endpoints} needs to be clear. {\em Builtin Endpoints} are predefined by \glref{RTPS} specification and once a {\em Participant} is discovered, it can be assumed that {\em Builtin Endpoints} are present, while {\em User-defined Endpoints} are defined by user application and it's purpose and can't be known in advance. Therefore the purpose of {\em Discovery Module} can be roughly simplified to discovering {\em User-defined Endpoints} of remote {\em Participants} with help of {\em Builtin Endpoints}. {\em Builtin Endpoints} are used by {\em Simple Discovery Protocols}.

\secc SPDP

{\em Best-effort Writer} with predefined Entity Id\fnote{ENTITYID\_SPDP\_BUILTIN\_PARTICIPANT\_WRITER} and {\em Best-effort Reader} with predefined Entity Id\fnote{ENTITYID\_SPDP\_BUILTIN\_PARTICIPANT\_READER} are used to exchange {\em SPDPdiscoveredParticipantData} containing {\em Participant Proxy} information about remote {\em Participant}. This traffic is sent to predefined \glref{IP} address and port discussed in \glref{PSM} in \cite[OMG:DDSI-RTPS22]. Remote {\em Participants} and their attributes are discovered by \glref{SPDP}.

\secc SEDP

{\em Reliable Writer} with predefined Entity Id\fnote{ENTITYID\_SEDP\_BUILTIN\_PUBLICATIONS\_WRITER} and {\em Reliable Reader} with predefined Entity Id\fnote{ENTITYID\_SEDP\_BUILTIN\_PUBLICATIONS\_READER} are used to exchange {\em DiscoveredWriterData} containing information about {\em Writers} of remote {\em Participant}.

{\em Reliable Writer} with predefined Entity Id\fnote{ENTITYID\_SEDP\_BUILTIN\_SUBSCRIPTIONS\_WRITER} and {\em Reliable Reader} with predefined Entity Id\fnote{ENTITYID\_SEDP\_BUILTIN\_SUBSCRIPTIONS\_READER} are used to exchange {\em DiscoveredReaderData} containing information about {\em Readers} of remote {\em Participant}.

The last pair of {\em Reliable Endpoints}\fnote{ENTITYID\_SEDP\_BUILTIN\_TOPIC\_WRITER and ENTITYID\_SEDP\_BUILTIN\_TOPIC\_READER} can be used to exchange {\em DiscoveredTopicData}, but these {\em Endpoints} aren't mandatory and the interoperability is not affected by them.

{\em Endpoints} of remote {\em Participants} are discovered by \glref{SEDP}.


%%%%%%%%%%%%%%%%%%%%%%%%%%%%%%%%%%%%%%%%%%%%%%%%%%%%%%%%%%%%%%%%%%%%%%%%%%%%%%%%
%%%%%%%%%%%%%%%%%%%%%%%%%%%%%% RTPS 1.0 %%%%%%%%%%%%%%%%%%%%%%%%%%%%%%%%%%%%%%%%
%%%%%%%%%%%%%%%%%%%%%%%%%%%%%%%%%%%%%%%%%%%%%%%%%%%%%%%%%%%%%%%%%%%%%%%%%%%%%%%%
\label[rtps10]\sec RTPS 1.0

\glref{ORTE} is one of the implementations of the \glref{RTPS} protocol used as proof of concept to standardize \glref{RTPS} 1.0. This section discuss the difference between version 1.0 and 2.2 of the \glref{RTPS} protocol, viewed from the perspective of the version 2.2.

\secc Structure Module

Important change in version 2.2 is that the {\em \glref{GUID}} consists of {\em Guid Prefix} (12B) and {\em Entity Id} (4B), while in version 1.0, {\em \glref{GUID}} consists of {\em Host Id} (4B), {\em Application Id} (4B) and {\em Object Id} (4B). {\em Entity Id} may be compared with {\em Object Id} and {\em Guid Prefix} corresponds to {\em Host Id} and {\em Application Id}. The size of {\em \glref{GUID}} in version 2.2 was increased by 4B.

New {\em Locator} type is introduced in version 2.2, containing \glref{IP} address, port and kind. {\em Locator} type is introduced because of \glref{IP}v6, it's kind can be either LOCATOR\_KIND\_UDPv4 or LOCATOR\_KIND\_UDPv6.

\secc Messages Module

The structure of the {\em \glref{RTPS} Header} remains same, but because of change in size of \glref{GUID} in version 2.2, size of the {\em Header} also increased. {\em Host Id} and {\em Application Id} are replaced by {\em Guid Prefix} indeed.

The structure of the {\em Submessage} remains completely same.

Following are changes in {\em Submessages}:
\begitems
* VAR is deprecated
* ISSUE is deprecated
* ACK is renamed to ACKNACK
* INFO\_REPLY is renamed to INFO\_REPLY\_IP4
* INFO\_REPLY is introduced
* NACK\_FRAG is introduced
* HEARTBEAT\_FRAG is introduced
* DATA is introduced
* DATA\_FRAG is introduced
\enditems

\secc Behavior Module

{\em Writers} are divided to {\em CSTWriters} and {\em Publications}, {\em Readers} to {\em CSTWriters} and {\em Subscriptions} in version 1.0.

{\em CSTWriters} and {\em CSTReaders} are builtin ``endpoints'' used for Composite State Transfer protocol and the communication between them is reliable. Messages used are VAR, GAP, HEARTBEAT and ACK. {\em CSChanges} are exchanges between {\em CSTWriters} and {\em CSTReadres}.

{\em Publications} and {\em Subscriptions} are used for user data exchange and the communication can be either reliable or best-effort. Messages used are ISSUE, HEARTBEAT and ACK. User data in ISSUE messages are represented in \glref{CDR} format.

In version 2.2, only difference between builtin and user defined {\em Endpoints} are predefined Entity Ids. Both, builtin and user defined {\em Endpoints} can be reliable or best-effort and DATA, HEARTBEAT, GAP and ACKNACK submessages are used independently on the purpose of the communication.

\secc Discovery Module

In version 1.0, two kinds of ``entities'' are discussed in specification - {\em Managers} and {\em Managed Applications}. The purpose of {\em Managers} is {\em Managed Application} discovery and the purpose of {\em Managed Applications} is to exchange user data. Each {\em Managed Application} needs to be registered to one of the {\em Managers}.

Discovery in version 1.0 is ensured by following protocols:
\begitems
* {\em Inter-Manager Protocol} allows {\em Managers} to discover each other.
* {\em Manager-Discovery Protocol} allows every {\em Managed Application} to discover other {\em Managers}.
* {\em Registration Protocol} allows {\em Managers} to find theirs {\em Managed Applications}.
* {\em Application-Discovery Protocol} is used by {\em Managed Application} to discover other {\em Managed Applications}.
* {\em Services-Discovery Protocol} allows to find {\em Publications} and {\em Subscriptions} in other {\em Managed Appliations}.
\enditems

In version 2.2, \glref{SPDP} and \glref{SEDP} are only required protocols used for discovery.

